% Options for packages loaded elsewhere
\PassOptionsToPackage{unicode}{hyperref}
\PassOptionsToPackage{hyphens}{url}
%
\documentclass[
]{article}
\usepackage{amsmath,amssymb}
\usepackage{iftex}
\ifPDFTeX
  \usepackage[T1]{fontenc}
  \usepackage[utf8]{inputenc}
  \usepackage{textcomp} % provide euro and other symbols
\else % if luatex or xetex
  \usepackage{unicode-math} % this also loads fontspec
  \defaultfontfeatures{Scale=MatchLowercase}
  \defaultfontfeatures[\rmfamily]{Ligatures=TeX,Scale=1}
\fi
\usepackage{lmodern}
\ifPDFTeX\else
  % xetex/luatex font selection
\fi
% Use upquote if available, for straight quotes in verbatim environments
\IfFileExists{upquote.sty}{\usepackage{upquote}}{}
\IfFileExists{microtype.sty}{% use microtype if available
  \usepackage[]{microtype}
  \UseMicrotypeSet[protrusion]{basicmath} % disable protrusion for tt fonts
}{}
\makeatletter
\@ifundefined{KOMAClassName}{% if non-KOMA class
  \IfFileExists{parskip.sty}{%
    \usepackage{parskip}
  }{% else
    \setlength{\parindent}{0pt}
    \setlength{\parskip}{6pt plus 2pt minus 1pt}}
}{% if KOMA class
  \KOMAoptions{parskip=half}}
\makeatother
\usepackage{xcolor}
\usepackage[margin=1in]{geometry}
\usepackage{color}
\usepackage{fancyvrb}
\newcommand{\VerbBar}{|}
\newcommand{\VERB}{\Verb[commandchars=\\\{\}]}
\DefineVerbatimEnvironment{Highlighting}{Verbatim}{commandchars=\\\{\}}
% Add ',fontsize=\small' for more characters per line
\usepackage{framed}
\definecolor{shadecolor}{RGB}{248,248,248}
\newenvironment{Shaded}{\begin{snugshade}}{\end{snugshade}}
\newcommand{\AlertTok}[1]{\textcolor[rgb]{0.94,0.16,0.16}{#1}}
\newcommand{\AnnotationTok}[1]{\textcolor[rgb]{0.56,0.35,0.01}{\textbf{\textit{#1}}}}
\newcommand{\AttributeTok}[1]{\textcolor[rgb]{0.13,0.29,0.53}{#1}}
\newcommand{\BaseNTok}[1]{\textcolor[rgb]{0.00,0.00,0.81}{#1}}
\newcommand{\BuiltInTok}[1]{#1}
\newcommand{\CharTok}[1]{\textcolor[rgb]{0.31,0.60,0.02}{#1}}
\newcommand{\CommentTok}[1]{\textcolor[rgb]{0.56,0.35,0.01}{\textit{#1}}}
\newcommand{\CommentVarTok}[1]{\textcolor[rgb]{0.56,0.35,0.01}{\textbf{\textit{#1}}}}
\newcommand{\ConstantTok}[1]{\textcolor[rgb]{0.56,0.35,0.01}{#1}}
\newcommand{\ControlFlowTok}[1]{\textcolor[rgb]{0.13,0.29,0.53}{\textbf{#1}}}
\newcommand{\DataTypeTok}[1]{\textcolor[rgb]{0.13,0.29,0.53}{#1}}
\newcommand{\DecValTok}[1]{\textcolor[rgb]{0.00,0.00,0.81}{#1}}
\newcommand{\DocumentationTok}[1]{\textcolor[rgb]{0.56,0.35,0.01}{\textbf{\textit{#1}}}}
\newcommand{\ErrorTok}[1]{\textcolor[rgb]{0.64,0.00,0.00}{\textbf{#1}}}
\newcommand{\ExtensionTok}[1]{#1}
\newcommand{\FloatTok}[1]{\textcolor[rgb]{0.00,0.00,0.81}{#1}}
\newcommand{\FunctionTok}[1]{\textcolor[rgb]{0.13,0.29,0.53}{\textbf{#1}}}
\newcommand{\ImportTok}[1]{#1}
\newcommand{\InformationTok}[1]{\textcolor[rgb]{0.56,0.35,0.01}{\textbf{\textit{#1}}}}
\newcommand{\KeywordTok}[1]{\textcolor[rgb]{0.13,0.29,0.53}{\textbf{#1}}}
\newcommand{\NormalTok}[1]{#1}
\newcommand{\OperatorTok}[1]{\textcolor[rgb]{0.81,0.36,0.00}{\textbf{#1}}}
\newcommand{\OtherTok}[1]{\textcolor[rgb]{0.56,0.35,0.01}{#1}}
\newcommand{\PreprocessorTok}[1]{\textcolor[rgb]{0.56,0.35,0.01}{\textit{#1}}}
\newcommand{\RegionMarkerTok}[1]{#1}
\newcommand{\SpecialCharTok}[1]{\textcolor[rgb]{0.81,0.36,0.00}{\textbf{#1}}}
\newcommand{\SpecialStringTok}[1]{\textcolor[rgb]{0.31,0.60,0.02}{#1}}
\newcommand{\StringTok}[1]{\textcolor[rgb]{0.31,0.60,0.02}{#1}}
\newcommand{\VariableTok}[1]{\textcolor[rgb]{0.00,0.00,0.00}{#1}}
\newcommand{\VerbatimStringTok}[1]{\textcolor[rgb]{0.31,0.60,0.02}{#1}}
\newcommand{\WarningTok}[1]{\textcolor[rgb]{0.56,0.35,0.01}{\textbf{\textit{#1}}}}
\usepackage{graphicx}
\makeatletter
\def\maxwidth{\ifdim\Gin@nat@width>\linewidth\linewidth\else\Gin@nat@width\fi}
\def\maxheight{\ifdim\Gin@nat@height>\textheight\textheight\else\Gin@nat@height\fi}
\makeatother
% Scale images if necessary, so that they will not overflow the page
% margins by default, and it is still possible to overwrite the defaults
% using explicit options in \includegraphics[width, height, ...]{}
\setkeys{Gin}{width=\maxwidth,height=\maxheight,keepaspectratio}
% Set default figure placement to htbp
\makeatletter
\def\fps@figure{htbp}
\makeatother
\setlength{\emergencystretch}{3em} % prevent overfull lines
\providecommand{\tightlist}{%
  \setlength{\itemsep}{0pt}\setlength{\parskip}{0pt}}
\setcounter{secnumdepth}{-\maxdimen} % remove section numbering
\ifLuaTeX
  \usepackage{selnolig}  % disable illegal ligatures
\fi
\usepackage{bookmark}
\IfFileExists{xurl.sty}{\usepackage{xurl}}{} % add URL line breaks if available
\urlstyle{same}
\hypersetup{
  pdftitle={CodingChallenge6},
  pdfauthor={Kylie Blake},
  hidelinks,
  pdfcreator={LaTeX via pandoc}}

\title{CodingChallenge6}
\author{Kylie Blake}
\date{2025-03-27}

\begin{document}
\maketitle

{
\setcounter{tocdepth}{2}
\tableofcontents
}
\section{Question 1}\label{question-1}

\textbf{1. 2 pts. Regarding reproducibility, what is the main point of
writing your own functions and iterations? }

\begin{itemize}
\tightlist
\item
  The main point of writing your own functions and iterations in ragards
  to reproducibility is that it reduces error from copy and pasting the
  same function or entering the same values multiple times. A custom
  function also improves consistency in your script and allows easier
  modifications to code if edits are needed.
\end{itemize}

\section{Question 2}\label{question-2}

\textbf{2. 2 pts. In your own words, describe how to write a function
and a for loop in R and how they work. Give me specifics like syntax,
where to write code, and how the results are returned.}

\begin{enumerate}
\def\labelenumi{\alph{enumi}.}
\tightlist
\item
  Writing a function:
\end{enumerate}

\begin{itemize}
\item
  To write a function, you first need to give the function a name,
  i.e.~NAME \textless- function(). This will store your function in the
  Environment.
\item
  Second, you need to give the function an input name(s) or argument(s),
  these arguments will be executed within the body of a function. Ex.
  function(velocity, time)
\item
  Lastly, the ``tasks'' of the function sit inside curly brackets, and
  are known as the body of the function. Ex. function(velocity, time)\{
  distance \textless- (velocity * time) return(distance) \}
\item
  To see the calculated distance in the console, you would tell the
  function to return() that variable.
\end{itemize}

\begin{enumerate}
\def\labelenumi{\alph{enumi}.}
\setcounter{enumi}{1}
\tightlist
\item
  Writing a for loop:
\end{enumerate}

In simplest terms, a for loop executes a task for every itemp in a
sequence, and continues running that task until the last value in the
sequence is reached.

\begin{itemize}
\item
  The first component of the loop is setting defining the sequence. The
  sequence may be numeric or characters. The syntax is always the same
  for(variable in vector). The variable can be named whatever name you
  prefer. For example ``for(i in 1:10)'' is saying for variable i in the
  sequence 1 through 10, or set variable i equal to 1:10.
\item
  Secondly, you need to give the for loop a task to do, which will sit
  in the body of the for loop. Again the body is represented with curly
  brackets. ex. for (i in 1:10) \{print(i+2)\}. This is saying take
  values 1 through 10 and add two to each value in the sequence.
\item
  For loops can become more complex by saving outputted values into a
  dataframe. In this situation, you would define a dataframe prior to
  executing the for loop with no values. ex. Table \textless- NULL. You
  would then define a dataframe for values within the body of the for
  loop.
\end{itemize}

\section{Question 3}\label{question-3}

\textbf{3. 2 pts. Read in the Cities.csv file from Canvas using a
relative file path.}

\begin{Shaded}
\begin{Highlighting}[]
\NormalTok{Cities }\OtherTok{\textless{}{-}} \FunctionTok{read.csv}\NormalTok{(}\StringTok{"CodingChallenge6/Cities.csv"}\NormalTok{)}
\end{Highlighting}
\end{Shaded}

\section{Question 4}\label{question-4}

\textbf{4. 6 pts. Write a function to calculate the distance between two
pairs of coordinates based on the Haversine formula (see below). The
input into the function should be lat1, lon1, lat2, and lon2. The
function should return the object distance\_km. All the code below needs
to go into the function.}

\begin{Shaded}
\begin{Highlighting}[]
\NormalTok{Function }\OtherTok{\textless{}{-}} \ControlFlowTok{function}\NormalTok{(lat1, lon1, lat2,lon2)\{}
\CommentTok{\# convert to radians}
\NormalTok{rad.lat1 }\OtherTok{\textless{}{-}}\NormalTok{ lat1 }\SpecialCharTok{*}\NormalTok{ pi}\SpecialCharTok{/}\DecValTok{180}
\NormalTok{rad.lon1 }\OtherTok{\textless{}{-}}\NormalTok{ lon1 }\SpecialCharTok{*}\NormalTok{ pi}\SpecialCharTok{/}\DecValTok{180}
\NormalTok{rad.lat2 }\OtherTok{\textless{}{-}}\NormalTok{ lat2 }\SpecialCharTok{*}\NormalTok{ pi}\SpecialCharTok{/}\DecValTok{180}
\NormalTok{rad.lon2 }\OtherTok{\textless{}{-}}\NormalTok{ lon2 }\SpecialCharTok{*}\NormalTok{ pi}\SpecialCharTok{/}\DecValTok{180}

\CommentTok{\# Haversine formula}
\NormalTok{delta\_lat }\OtherTok{\textless{}{-}}\NormalTok{ rad.lat2 }\SpecialCharTok{{-}}\NormalTok{ rad.lat1}
\NormalTok{delta\_lon }\OtherTok{\textless{}{-}}\NormalTok{ rad.lon2 }\SpecialCharTok{{-}}\NormalTok{ rad.lon1}
\NormalTok{a }\OtherTok{\textless{}{-}} \FunctionTok{sin}\NormalTok{(delta\_lat }\SpecialCharTok{/} \DecValTok{2}\NormalTok{)}\SpecialCharTok{\^{}}\DecValTok{2} \SpecialCharTok{+} \FunctionTok{cos}\NormalTok{(rad.lat1) }\SpecialCharTok{*} \FunctionTok{cos}\NormalTok{(rad.lat2) }\SpecialCharTok{*} \FunctionTok{sin}\NormalTok{(delta\_lon }\SpecialCharTok{/} \DecValTok{2}\NormalTok{)}\SpecialCharTok{\^{}}\DecValTok{2}
\NormalTok{c }\OtherTok{\textless{}{-}} \DecValTok{2} \SpecialCharTok{*} \FunctionTok{asin}\NormalTok{(}\FunctionTok{sqrt}\NormalTok{(a)) }

\CommentTok{\# Earth\textquotesingle{}s radius in kilometers}
\NormalTok{earth\_radius }\OtherTok{\textless{}{-}} \DecValTok{6378137}

\CommentTok{\# Calculate the distance}
\NormalTok{distance\_km }\OtherTok{\textless{}{-}}\NormalTok{ (earth\_radius }\SpecialCharTok{*}\NormalTok{ c)}\SpecialCharTok{/}\DecValTok{1000}

\FunctionTok{return}\NormalTok{(distance\_km) \}}
\end{Highlighting}
\end{Shaded}

\section{Question 5}\label{question-5}

\textbf{5. 5 pts. Using your function, compute the distance between
Auburn, AL and New York City }

\textbf{a. Subset/filter the Cities.csv data to include only the
latitude and longitude values you need and input as input to your
function.}

\textbf{b. The output of your function should be 1367.854 km}

\begin{Shaded}
\begin{Highlighting}[]
\CommentTok{\#Part a}
\CommentTok{\#Subset out nyc data and Auburn data}
\NormalTok{nyc }\OtherTok{\textless{}{-}} \FunctionTok{subset}\NormalTok{(Cities, city }\SpecialCharTok{==} \StringTok{"New York"}\NormalTok{)}
\NormalTok{auburn }\OtherTok{\textless{}{-}} \FunctionTok{subset}\NormalTok{(Cities, city }\SpecialCharTok{==} \StringTok{"Auburn"}\NormalTok{)}

\NormalTok{lat1 }\OtherTok{\textless{}{-}}\NormalTok{ nyc}\SpecialCharTok{$}\NormalTok{lat}
\NormalTok{lon1 }\OtherTok{\textless{}{-}}\NormalTok{ nyc}\SpecialCharTok{$}\NormalTok{long}
\NormalTok{lat2 }\OtherTok{\textless{}{-}}\NormalTok{ auburn}\SpecialCharTok{$}\NormalTok{lat}
\NormalTok{lon2 }\OtherTok{\textless{}{-}}\NormalTok{ auburn}\SpecialCharTok{$}\NormalTok{long}

\CommentTok{\#Part b {-} Calculate distance from NYC to Auburn}
\NormalTok{distance\_km }\OtherTok{\textless{}{-}} \FunctionTok{Function}\NormalTok{(lat1, lon1, lat2, lon2)}
\FunctionTok{print}\NormalTok{(distance\_km)}
\end{Highlighting}
\end{Shaded}

\begin{verbatim}
## [1] 1367.854
\end{verbatim}

\section{Question 6}\label{question-6}

\textbf{6. 6 pts. Now, use your function within a for loop to calculate
the distance between all other cities in the data. The output of the
first 9 iterations is shown below. }

\begin{Shaded}
\begin{Highlighting}[]
\CommentTok{\#list city names }
\NormalTok{city\_name }\OtherTok{\textless{}{-}}\NormalTok{ Cities}\SpecialCharTok{$}\NormalTok{city}
\NormalTok{city\_name}
\end{Highlighting}
\end{Shaded}

\begin{verbatim}
##  [1] "New York"      "Los Angeles"   "Chicago"       "Miami"        
##  [5] "Houston"       "Dallas"        "Philadelphia"  "Atlanta"      
##  [9] "Washington"    "Boston"        "Phoenix"       "Detroit"      
## [13] "Seattle"       "San Francisco" "San Diego"     "Minneapolis"  
## [17] "Tampa"         "Brooklyn"      "Denver"        "Queens"       
## [21] "Riverside"     "Las Vegas"     "Baltimore"     "St. Louis"    
## [25] "Portland"      "San Antonio"   "Sacramento"    "Austin"       
## [29] "Orlando"       "San Juan"      "San Jose"      "Indianapolis" 
## [33] "Pittsburgh"    "Cincinnati"    "Manhattan"     "Kansas City"  
## [37] "Cleveland"     "Columbus"      "Bronx"         "Auburn"
\end{verbatim}

\begin{Shaded}
\begin{Highlighting}[]
\ControlFlowTok{for}\NormalTok{(i }\ControlFlowTok{in} \FunctionTok{seq\_along}\NormalTok{(city\_name))\{}
\NormalTok{all\_cities }\OtherTok{\textless{}{-}} \FunctionTok{subset}\NormalTok{(Cities, city }\SpecialCharTok{==}\NormalTok{ city\_name[i]) }\CommentTok{\#Go through all city names for i}
\NormalTok{  lat1 }\OtherTok{\textless{}{-}}\NormalTok{ all\_cities}\SpecialCharTok{$}\NormalTok{lat }
\NormalTok{  lon1 }\OtherTok{\textless{}{-}}\NormalTok{ all\_cities}\SpecialCharTok{$}\NormalTok{long}
\NormalTok{  lat2 }\OtherTok{\textless{}{-}}\NormalTok{ auburn}\SpecialCharTok{$}\NormalTok{lat}
\NormalTok{  lon2 }\OtherTok{\textless{}{-}}\NormalTok{ auburn}\SpecialCharTok{$}\NormalTok{long}
  
\NormalTok{distance\_km }\OtherTok{\textless{}{-}} \FunctionTok{Function}\NormalTok{(lat1, lon1, lat2, lon2)}
\FunctionTok{print}\NormalTok{(distance\_km)\}}
\end{Highlighting}
\end{Shaded}

\begin{verbatim}
## [1] 1367.854
## [1] 3051.838
## [1] 1045.521
## [1] 916.4138
## [1] 993.0298
## [1] 1056.022
## [1] 1239.973
## [1] 162.5121
## [1] 1036.99
## [1] 1665.699
## [1] 2476.255
## [1] 1108.229
## [1] 3507.959
## [1] 3388.366
## [1] 2951.382
## [1] 1530.2
## [1] 591.1181
## [1] 1363.207
## [1] 1909.79
## [1] 1380.138
## [1] 2961.12
## [1] 2752.814
## [1] 1092.259
## [1] 796.7541
## [1] 3479.538
## [1] 1290.549
## [1] 3301.992
## [1] 1191.666
## [1] 608.2035
## [1] 2504.631
## [1] 3337.278
## [1] 800.1452
## [1] 1001.088
## [1] 732.5906
## [1] 1371.163
## [1] 1091.897
## [1] 1043.273
## [1] 851.3423
## [1] 1382.372
## [1] 0
\end{verbatim}

\section{Bonus}\label{bonus}

\textbf{Bonus point if you can have the output of each iteration append
a new row to a dataframe, generating a new column of data. In other
words, the loop should create a dataframe with three columns called
city1, city2, and distance\_km, as shown below. The first six rows of
the dataframe are shown below.}

\begin{itemize}
\tightlist
\item
  I could not figure this out! If we could break it down in class, I
  would appreciate that!
\end{itemize}

\section{Question 7}\label{question-7}

\textbf{7. 2 pts. Commit and push a gfm .md file to GitHub inside a
directory called Coding Challenge 6. Provide me a link to your github
written as a clickable link in your .pdf or .docx}

\emph{\href{https://github.com/kzb0180}{Click here to my GitHub}}

\end{document}
